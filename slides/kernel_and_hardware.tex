\section{Hardware and Kernel}
\label{sec:kernel}

\subsection{Some hardware considerations}
\label{subsec:more_hard}



\begin{frame}
  \frametitle{Processing architectures}

  \begin{itemize}
  \item Mono-processor (a.k.a The good ol' days)
    \begin{itemize}
    \item One CPU, One memory bus
    \item Time-slicing
    \end{itemize}
  \item Multi-processor
    \begin{itemize}
    \item SMP, Symmetric Multi-Processing
    \item Many CPUs, One Memory bus
    \end{itemize}
  \item NUMA
    \begin{itemize}
    \item Memory bus is the bottleneck
    \item Many CPUs, Many memory bus
    \item One shared memory bus
    \end{itemize}
  \end{itemize}
\end{frame}


\begin{frame}
  \frametitle{Caches}

  \begin{itemize}
  \item No direct memory accesses
  \item Always go through caches
  \item \emph{Write-through} vs \emph{Write-back}
  \item \emph{Inclusive} vs \emph{exclusive}
  \item Different levels
    \begin{itemize}
    \item L1: 2 per-core (data/instructions)
    \item L2: shared between some cores
    \item L3: Shared between all cores
    \end{itemize}
  \end{itemize}
\end{frame}


\begin{frame}
  \frametitle{Cache lines}

  \begin{itemize}
  \item Memory caching granularity
    \begin{itemize}
    \item Usually 64 bytes \emph{(8 * int64\_t)}
    \end{itemize}
  \item Cache associativity
    \begin{itemize}
    \item Direct-mapping: 1 possible line for every memory address
    \item n-way associative: n possible lines
    \item Fully associative: Any line can cache anything
    \item Modern CPUs: n-way associative (L1: 2/4, L2: 8/16)
    \end{itemize}
  \item Line sharing and MESI Protocol
    \begin{itemize}
    \item Modified, Exclusive, Shared, Invalid
    \end{itemize}
  \end{itemize}
\end{frame}


\begin{frame}
  \frametitle{Memory barriers}

  \begin{itemize}
  \item Cache invalidation is \emph{``slow''}
  \item $\rightarrow$ Done asynchronously
    \begin{itemize}
    \item Store buffer
    \item Invalidation queue
    \end{itemize}
  \item Operation on a CPU takes time to propagate
  \item \textbf{-\_-}
  \item Memory barriers are ``instructions'' which synchronously flush the buffers
  \item Also prevent some instruction reordering
  \item Very important with atomic operations
  \end{itemize}
\end{frame}


\begin{frame}
  \frametitle{Pre-fetching and branch prediction}

  \begin{itemize}
  \item Access to memory is slow
  \item Instruction execution is fast
  \item Let's preload the program instructions
  \item Lots of jump
    \begin{itemize}
    \item Control statements
    \item Function calls
    \end{itemize}
  \item The compiler and CPU tries to predict jumps
  \end{itemize}
\end{frame}


\begin{frame}
  \frametitle{Out of order execution}

  \begin{itemize}
  \item What you see is NOT what you get
    \begin{itemize}
    \item Compiler optimizations
    \item Processor optimizations !
    \end{itemize}
  \item Instructions are reorganized
  \item \emph{Out of order execution} or \emph{dynamic instruction scheduling}
    \begin{itemize}
    \item Instruction queue
    \item Executed based on data availability
    \end{itemize}
  \item Superscalar architecture
    \begin{itemize}
    \item Instruction level parallelism
    \item Use of multiple execution unit
    \end{itemize}
  \end{itemize}
\end{frame}


\begin{frame}
  \frametitle{SMT}

  \begin{itemize}
  \item Simultaneous Multi Threading
  \item A.k.a \emph{HyperThreading}
  \item 2 (or more) threads on one core.
  \item Increase usage of
    \begin{itemize}
    \item Dynamic instruction scheduling
    \item Super-scalar
    \end{itemize}
  \item 2 logical core, only one real core
  \item 0-25\% performance gain
  \item Can reduce performance !
  \end{itemize}
\end{frame}

\subsection{Scheduling}
\label{subsec:sched}


\begin{frame}
  \frametitle{Tasks, threads and processes}

  \begin{itemize}
  \item In the kernel, everything is a \emph{task}\footnote{task\_struct}
    \begin{itemize}
    \item Execution context that can be allocated CPU time
      \begin{itemize}
      \item Something \emph{schedulable}
      \end{itemize}
    \item With associated resources: Memory\footnote{mm\_struct}, Files\footnote{files\_struct}, ...
    \end{itemize}
  \item Used for userland and kernel
    \begin{itemize}
    \item Userland task $\rightarrow$ thread
    \item Threads w/ shared memory and fds $\rightarrow$ process
    \end{itemize}
  \item One of the 3 kernel's core concepts
  \end{itemize}
\end{frame}


\begin{frame}
  \frametitle{What is scheduling ?}

  \begin{itemize}
  \item The art of sharing the CPU(s)
  \item NP-complete problem
  \item Kernel's role
  \item What is running ? Where ? For how long ?
  \item Are all the tasks equals ?
  \item What is the cost ?
  \end{itemize}
\end{frame}


\begin{frame}
  \frametitle{The cost of scheduling}

  \begin{itemize}
  \item Context switching
    \begin{itemize}
    \item To the kernel
    \item To another process
    \end{itemize}
  \item Latency vs throughput
  \item When to schedule is important
    \begin{itemize}
    \item Suspend a task holding mutexes ?
    \item Before an I/O operation ?
    \end{itemize}
  \end{itemize}
\end{frame}


\begin{frame}
  \frametitle{When does scheduling happen ?}

  \begin{itemize}
  \item When a thread waits (I/O, mutex, ...)
  \item When returning to user-space from a syscalls
  \item When returning from an interrupt
    \begin{itemize}
    \item Device event (keyboard, network, ...)
    \item Timer event
    \end{itemize}
  \end{itemize}
\end{frame}


\begin{frame}
  \frametitle{Scheduling classes}

  \begin{itemize}
  \item What are scheduling classes:
    \begin{itemize}
    \item Different algorithms
    \item Different priority
    \end{itemize}
  \item The 'normal' classes:
    \begin{itemize}
    \item \emph{SCHED\_NORMAL} (a.k.a \emph{SCHED\_OTHER} in userland)
    \item \emph{SCHED\_BATCH}. \emph{SCHED\_NORMAL} for CPU-intensive tasks
    \item \emph{SCHED\_IDLE}. Very low priority. Only run when CPU would idle.
    \end{itemize}
  \item The 'real-time' classes:
    \begin{itemize}
    \item \emph{SCHED\_FIFO}
    \item \emph{SCHED\_RR}
    \item \emph{SCHED\_DEADLINE}
    \end{itemize}
  \end{itemize}
\end{frame}


\begin{frame}
  \frametitle{CFS}

  \begin{itemize}
  \item The Completely Fair Scheduler\footnote{Since 2.6.23}
  \item Current scheduler for \emph{SCHED\_NORMAL}
  \item Models an \emph{ideal} CPU, perfectly shared
  \item CPU usage is tracked with nano-second precision
  \item rbtree / O(log n)
  \item \emph{Sleeper fairness}
  \end{itemize}
\end{frame}

\begin{frame}
  \frametitle{Other features}

  \begin{itemize}
  \item Process priorities (aka \emph{niceness})
    \begin{itemize}
    \item between process of the same class
    \end{itemize}
  \item Processor affinity
  \end{itemize}
\end{frame}


\begin{frame}
  \frametitle{Spinlock, mutex, futex}

  \begin{itemize}
  \item The basic locking primitive is the spinlock
    \begin{itemize}
    \item Waste CPU cycles until ready
    \item Atomic variable
    \end{itemize}
  \item Mutex = spinlock + scheduler cooperation
  \item Futex\footnote{man 2/7 futex} stands for Fast Userland muTEX
    \begin{itemize}
    \item Library and syscalls to implement \emph{pthread\_mutex\_xxx()}
    \item No syscall when no contention
    \end{itemize}
  \end{itemize}
\end{frame}

\subsection{Syscall and real-time}
\label{subsec:rt}


\begin{frame}
  \frametitle{What's a syscall ?}

  \begin{itemize}
  \item The kernel API
    \begin{itemize}
    \item i.e: The only way for the userland to talk to the kernel
    \end{itemize}
  \item Use processor's software interrupt mechanism (\emph{int 80})
    \begin{itemize}
    \item Context-switch to a predefined kernel handler
    \item Back to ring 0
    \item The kernel does its thing
    \item Context-switch back to the calling userland thread
    \end{itemize}
  \item No way for the user thread to resume until completion
  \item Preemption point
  \end{itemize}
\end{frame}

\begin{frame}
  \frametitle{Real time programming}

  \begin{itemize}
  \item Deterministic programming
  \item Response time guarantees (deadlines)
  \item Different types:
    \begin{itemize}
    \item Hard: Airplanes, nuclear reactor, car control
    \item Firm: Sound on your computer
    \item Soft: Audio/Video streaming
    \end{itemize}
  \end{itemize}
\end{frame}


\begin{frame}
  \frametitle{Worst case scenario}

  \begin{itemize}
  \item RT system is defined by it's worst-case
  \item Investigate worst-case execution time.
    \begin{itemize}
    \item Is it known ?
    \item Is it good enough ?
    \item Can you control it ?
    \end{itemize}
  \item Almost everything is optimized for the average-case
  \end{itemize}
\end{frame}


\begin{frame}
  \frametitle{Practically ?}

  \begin{itemize}
  \item No (\emph{blocking}) syscalls
    \begin{itemize}
    \item Goodbye \emph{read()} and \emph{write()}
    \item You need to know about the kernel's internals
    \end{itemize}
  \item Or libc functions which are doing syscalls
    \begin{itemize}
    \item Goodbye \emph{malloc()} and \emph{free()}
    \item ...
    \end{itemize}
  \item Synchronization is critical
    \begin{itemize}
    \item Goodbye \emph{mutex\_lock()}
    \item Hello lock-free data structures
    \end{itemize}
  \item Takes the rest of the system into account
  \item Evaluate the worst case complexity
  \end{itemize}
\end{frame}


\begin{frame}
  \frametitle{Why do we care about real-time ?}

  \begin{itemize}
  \item Make us think about
    \begin{itemize}
    \item Latency
    \item p99
    \end{itemize}
  \item The field pioneered:
    \begin{itemize}
    \item Asynchronous I/Os
    \item Lock-free programming
    \item ...
    \end{itemize}
  \item What would you do if you \textbf{had} to have a strict latency
  \end{itemize}
\end{frame}



\subsection{Memory management}
\label{subsec:memory}


\begin{frame}
  \frametitle{Virtual memory (bis)}

  \begin{itemize}
  \item Controls the MMU
    \begin{itemize}
    \item Page directory (mapping between real/virtual addresses)
    \item Translation Lookaside Buffer
    \end{itemize}
  \item Swap
  \item Pretty complex system
    \begin{itemize}
    \item Kernel's own needs
    \item User memory
    \item Lots of sharing
      \begin{itemize}
      \item Threads: heap vs stack
      \item SYSV/POSIX shared memory
      \item \emph{mmpa()}
      \item DMA
      \end{itemize}
    \end{itemize}
  \item Memory sharing (SYSV, threads, mmap, ...)
  \end{itemize}
\end{frame}


\begin{frame}
  \frametitle{Memory pressure}

  \begin{itemize}
  \item Balance between different memory users
    \begin{itemize}
    \item Userland programs
    \item Drivers
    \item Various caches (fs, network, ...)
    \end{itemize}
  \item What happens if there are no memory:
    \begin{itemize}
    \item Disk write ?
    \item User request ?
    \item Interrupt handler ?
    \end{itemize}
  \end{itemize}
\end{frame}


\begin{frame}
  \frametitle{NUMA}

  \begin{itemize}
  \item Memory access from a different NUMA node is slow
  \item Each NUMA-node is managed separately
  \item Scheduler is NUMA-aware
    \begin{itemize}
    \item It tries to keep a task on the same CPU
    \end{itemize}
  \item Memory allocation is NUMA-aware
    \begin{itemize}
    \item It tries to allocate on the local NUMA-node
    \end{itemize}
  \item User can help:
    \begin{itemize}
    \item \emph{libnuma} and \emph{numactl}
    \item \emph{set\_mempolicy()}, ...
    \item \emph{sched\_setaffinity()}
    \end{itemize}
  \end{itemize}
\end{frame}


\begin{frame}[fragile]
  \frametitle{Swapping}

  \begin{itemize}
  \item Use disk as memory
  \item \textbf{Very} slow
    \begin{itemize}
    \item Memory access: ~ 100 ns
    \item Disk access: ~ 1 ms = 1 000 000 ns
    \end{itemize}
  \item Should never happen...
  \item ... but it's better than a failed allocation
  \item \verb+echo -15 > /proc/2592/oom_adj+
  \item man 2 \emph{mlock()}
  \end{itemize}
\end{frame}


\subsection{Errors and Performance}
\label{subsec:perf}

\begin{frame}
  \frametitle{Classical problems}

  \begin{itemize}
  \item Deadlock
  \item Live lock
  \item Priority inversion
    \begin{itemize}
    \item Priority inheritance: \emph{pthread\_mutexattr\_setprotocol()}
    \end{itemize}
  \item Convoying
  \item Signal handling
    \begin{itemize}
    \item No mutex allowed
    \end{itemize}
  \end{itemize}
\end{frame}


\begin{frame}
  \frametitle{Lock contention}

  \begin{itemize}
  \item Time spent waiting on lock ?
  \item Measure
  \item Re-design
  \item Removes locks
    \begin{itemize}
    \item Split data (differently)
    \item DCLP: Double Check Locking Pattern
    \item Lock-free
    \end{itemize}
  \end{itemize}
\end{frame}

\begin{frame}
  \frametitle{Cache-fighting}

  \begin{itemize}
  \item Don't forget about cache
    \begin{itemize}
    \item Lines
    \item Associativity
    \item Sharing
    \item Coherence
    \end{itemize}
  \item Caches are fighting when accessing the same data
  \item Split the data
  \item Duplicate if necessary
  \end{itemize}
\end{frame}

\begin{frame}
  \frametitle{False sharing}

  \begin{itemize}
  \item Cache-fighting for different data
  \item on the same cache lines !
  \item Split/duplicate the data
  \item Add some padding
  \item Use a cache-aware memory allocator
  \end{itemize}
\end{frame}
